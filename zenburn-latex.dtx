% \iffalse meta-comment
%
% Copyright (c) 2019 by Keith F. Prussing <kprussing74@gmail.com>
%
% This work may be distributed and/or modified under the conditions of
% the LaTeX Project Public License, either version 1.3c of this license
% or (at your option) any later version.  The latest version of this
% license is in
%
%     http://www.latex-project.org/lppl.txt
%
% and version 1.3c or later is part of all distributions of LaTeX
% version 2005/12/01 or later.
%
% This work has the LPPL maintenance status `maintained'.
%
% The Current Maintainer of this work is Keith F. Prussing.
%
% \fi
%
% \iffalse
%<*driver>
\ProvidesFile{\jobname.dtx}
%</driver>
%<common>\NeedsTeXFormat{LaTeX2e}[2003/12/01]
%<style>\ProvidesPackage{zenburn}
%<beamercolor>\ProvidesPackage{beamercolorthemezenburn}
%<*common>
    [2019/01/14 v1.0 Zenburn color themes]
%</common>
%
%<*driver>
\documentclass{ltxdoc}
\usepackage{zenburn}
\EnableCrossrefs{}
\CodelineIndex{}
\RecordChanges{}
\usepackage{hologo}
\lstset{%
    language=[08]Fortran,
    style=zenburn,
    showspaces=false,
    framerule=0pt,
    stepnumber=2,
    numbers=left,
    numbersep=5pt,
}
\usepackage[columns=1]{idxlayout}
\usepackage[style=ieee]{biblatex}
\usepackage{cleveref}
\addbibresource{references.bib}
\begin{document}
    \DocInput{\jobname.dtx}
\end{document}
%</driver>
% \fi
%
% \CheckSum{0}
%
% \CharacterTable
%  {Upper-case \A\B\C\D\E\F\G\H\I\J\K\L\M\N\O\P\Q\R\S\T\U\V\W\X\Y\Z
%   Lower-case \a\b\c\d\e\f\g\h\i\j\k\l\m\n\o\p\q\r\s\t\u\v\w\x\y\z
%   Digits     \0\1\2\3\4\5\6\7\8\9
%   Exclamation   \!     Double quote  \"     Hash (number) \#
%   Dollar        \$     Percent       \%     Ampersand     \&
%   Acute accent  \'     Left paren    \(     Right paren   \)
%   Asterisk      \*     Plus          \+     Comma         \,
%   Minus         \-     Point         \.     Solidus       \/
%   Colon         \:     Semicolon     \;     Less than     \<
%   Equals        \=     Greater than  \>     Question mark \?
%   Commercial at \@     Left bracket  \[     Backslash     \\
%   Right bracket \]     Circumflex    \^     Underscore    \_
%   Grave accent  \`     Left brace    \{     Vertical bar  \|
%   Right brace   \}     Tilde         \~}
%
% \changes{v1.0}{2019/01/14}{Initial version}
% \GetFileInfo{\jobname.dtx}
%
%^^A Package macros
% \DoNotIndex{\begin, \end, \newenvironment, \newcommand, \renewcommand}
% \DoNotIndex{\CurrentOption, \DeclareOption, \ExecuteOptions}
% \DoNotIndex{\PackageError, \PackageWarning, \PassOptionsToPackage}
% \DoNotIndex{\ProcessOptions, \RequirePackage, \relax}
%^^A Other package macros
% \DoNotIndex{\BeforeBeginEnvironment, \AfterEndEnvironment}
% \DoNotIndex{\color, \colorlet, \definecolor, \nopagecolor}
% \DoNotIndex{\lstdefinestyle, \lstset}
% \DoNotIndex{\iftoggle, \newtoggle, \togglefalse, \toggletrue}
%^^A Font macros
% \DoNotIndex{\bfseries, \itshape, \ttfamily, \small, \tiny}
%^^A Beamer macros
% \DoNotIndex{\DeclareOptionBeamer, \ExecuteOptionsBeamer}
% \DoNotIndex{\ProcessOptionsBeamer, \setbeamercolor, \usebeamercolor}
% \DoNotIndex{\mode}
%
% \title{Zenburn \LaTeX{} styles \thanks{This document
% corresponds to \textsf{\jobname}~\fileversion, dated~\filedate.}}
%
% \author{Keith F. Prussing \\ \texttt{kprussing74@gmail.com}}
%
% \maketitle
%
% \begin{abstract}
% An adaptation of the Zenburn color theme for vim for use in \LaTeX{}.
% \end{abstract}
%
% \section{Introduction}
%
% This package is a translation of the Zenburn color scheme for
% Vim~\cite{jnurmine_zenburn_2019} to be used within a \LaTeX{}
% document.  It defines the base colors, provides a style for use with
% \textsf{listings}, and a color theme for \textsf{beamer}.
%
% The package not only defines the true Zenburn dark theme, but it also
% provides a light theme derived from reversing the foreground and
% background of the normal text.  The colors are adjusted to ensure the
% contrast is sufficient to be legible.  Only the color needed by
% \textsf{listings} are defined with a ``dark'' and a ``light'' version.
% A subset of the colors in the full Zenburn theme have been defined
% as \textsf{xcolor} |\color|s.  These are listed in~\cref{sec:colors}.
% For convenience, the theme colors are also mapped to colors with the
% spaces replaced by hyphens.
%
% \DescribeMacro{zenburn dark style}
% \DescribeMacro{zenburn light style}
% \DescribeMacro{zenburn}
% A dark and a light |style| are provided for use with |\lstset| and
% family.  These styles set the colors and the font families for the
% language.  The style appropriate style is mapped to the |zenburn|
% style as well.
%
% \DescribeEnv{zenburn dark code block}
% \DescribeEnv{zenburn light code block}
% \DescribeEnv{code block}
% This package also provides environments to place a |lstlisting|
% environment in a |tcolorbox|.  This works around the fact that the
% frames defined by \textsf{listings} do not always fill the background.
% The rationale is: It is better to let a package that is good at
% working with frames handle frames.  Unlike the colors and styles, the
% generic environment provided does \emph{not} use the |zenburn| token.
% This is because you will most likely simply want to use a code block
% if you are loading this package.  As an aide to writing code snippets,
% this package wraps the |lstlisting| environment in a default
% |code block| block environment.
%
% The \textsf{beamer} color theme takes a reverse approach.  The default
% slide theme is dark text on a light background with the default dark
% code style.  The ``dark'' theme is the reverse.  The rationale is you
% will most likely want the code snippets in the proper Zenburn color
% scheme as opposed to the light version.  The theme sets the palettes
% and relies on the other themes to define the rest of the presentation.
%
% \section{Usage}
%
% To install the package, simply run the provided |zenburn-latex.ins|
% file through your preferred \LaTeX{}.  Then, copy the resultant |.sty|
% files to a location where \LaTeX{} can find them.  To generate the
% documentation, run this file (|zenburn-latex.dtx|) through
% \hologo{pdfLaTeX} twice.  The second time is only to make sure the
% references and index are typeset correctly.  Using the theme or
% package is simply a matter of adding |\usepackage{zenburn}| or
% |\usecolortheme{zenburn}| to your preamble.
%
% The better way to build the files is to simply use \textsf{SCons} with
% the provided |SConstruct|.
%
% \section{Example usage}
%
% \iffalse
%<*example>
% \fi
\begin{lstlisting}
program hello
    ! A simple example program in a forced code block
    use, intrinsic :: iso_fortran_env, only: output_unit
    implicit none
    write(output_unit, *) "Hello world"
end program hello
\end{lstlisting}

\begin{zenburn dark code block}
\begin{lstlisting}
program hello
    ! A simple example program in a dark code block
    use, intrinsic :: iso_fortran_env, only: output_unit
    implicit none
    write(output_unit, *) "Hello world"
end program hello
\end{lstlisting}
\end{zenburn dark code block}

\begin{zenburn light code block}
\begin{lstlisting}
program hello
    ! A simple example program in a light code block
    use, intrinsic :: iso_fortran_env, only: output_unit
    implicit none
    write(output_unit, *) "Hello world"
end program hello
\end{lstlisting}
\end{zenburn light code block}

\begin{code block}
\begin{lstlisting}
program hello
    ! A simple example program in a default code block
    use, intrinsic :: iso_fortran_env, only: output_unit
    implicit none
    write(output_unit, *) "Hello world"
end program hello
\end{lstlisting}
\end{code block}

%^^A The following should stop with an error about nested code blocks.
%^^A\begin{code block}
%^^A\begin{zenburn dark code block}
%^^A\begin{lstlisting}
%^^Aprogram hello
%^^A    ! A simple example program in a nested code block
%^^A    use, intrinsic :: iso_fortran_env, only: output_unit
%^^A    implicit none
%^^A    write(output_unit, *) "Hello world"
%^^Aend program hello
%^^A\end{lstlisting}
%^^A\end{zenburn dark code block}
%^^A\end{code block}
% \iffalse
%</example>
% \fi
%
% \StopEventually{
% \printbibliography[]
% \PrintChanges{}
% \PrintIndex{}
%}
%
% \section{Implementation}
%
% \subsection{Style}
%
% The style file defines the colors and defines a listing style.
%
% \subsubsection{Color definitions\label{sec:colors}}
%
% \iffalse
%<*style>
% \fi
% We begin by defining the colors in the theme.  Each color is a
% descriptive name prefixed with |zenburn |.  Note, we use spaces in the
% ``true'' color names.
%    \begin{macrocode}
\RequirePackage{xcolor}
\definecolor{zenburn foreground}        {HTML}{DCDCCC}
\definecolor{zenburn background}        {HTML}{3F3F3F}
\definecolor{zenburn character}         {HTML}{DCA3A3}
\definecolor{zenburn conditional}       {HTML}{F0DFAF}
\definecolor{zenburn comment}           {HTML}{7F9F7F}
\definecolor{zenburn debug}             {HTML}{BCA3A3}
\definecolor{zenburn delimiter}         {HTML}{8F8F8F}
\definecolor{zenburn diff add}          {HTML}{709080}
\definecolor{zenburn error message}     {HTML}{80d4AA}
\definecolor{zenburn exception}         {HTML}{C3BF9F}
\definecolor{zenburn keyword}           {HTML}{F0DFAF}
\definecolor{zenburn line number}       {HTML}{9FAFAF}
\definecolor{zenburn pmenu}             {HTML}{242424}
\definecolor{zenburn number}            {HTML}{8CD0D3}
\definecolor{zenburn special char}      {HTML}{DCA3a3}
\definecolor{zenburn special comment}   {HTML}{82A282}
\definecolor{zenburn special key}       {HTML}{9ECE9E}
\definecolor{zenburn statement}         {HTML}{E3CEAB}
\definecolor{zenburn string}            {HTML}{CC9393}
%    \end{macrocode}
%
% Now we define the color mappings for the dark and light color schemes.
% The dark color scheme is easy.  We just map back to the proper colors.
%    \begin{macrocode}
\colorlet{zenburn dark foreground}  {zenburn foreground}
\colorlet{zenburn dark background}  {zenburn background}
\colorlet{zenburn dark comment}     {zenburn comment}
\colorlet{zenburn dark keyword}     {zenburn keyword}
\colorlet{zenburn dark line number} {zenburn line number}
\colorlet{zenburn dark string}      {zenburn string}
%    \end{macrocode}
%
% The light color scheme takes a bit more work.  We start by reversing
% the normal text and lightening the background.  After that, we have to
% go through some trial and error to find colors that keep the feel of
% the dark theme and have enough contrast to be legible.
%    \begin{macrocode}
\colorlet{zenburn light foreground} {zenburn background}
\colorlet{zenburn light background} {zenburn foreground!30!white}
\colorlet{zenburn light comment}    {zenburn comment}
\colorlet{zenburn light keyword}    {zenburn keyword!70!black} %^^A or statement
\colorlet{zenburn light line number}{zenburn line number}
\colorlet{zenburn light string}     {zenburn string}
%    \end{macrocode}
%
% \iffalse
%</style>
% \fi
%
% \subsubsection{\textsf{Listing} styles}
%
% \iffalse
%<*style>
% \fi
% \begin{macro}{zenburn dark style}
% Next, we map our defined colors to the \textsf{listings} style.  We
% define both a light and dark style along with the remaining
% properties.
%    \begin{macrocode}
\RequirePackage{listings}
\lstdefinestyle{zenburn dark style}{%
    basicstyle=\small\ttfamily\color{zenburn dark foreground},
    keywordstyle=\bfseries\color{zenburn dark keyword},
    commentstyle=\itshape\color{zenburn dark comment},
    stringstyle=\color{zenburn dark string},
    showstringspaces=false,
    numberstyle=\tiny\color{zenburn dark line number},
}
%    \end{macrocode}
% \end{macro}
%
% \begin{macro}{zenburn dark style}
%    \begin{macrocode}
\RequirePackage{listings}
\lstdefinestyle{zenburn light style}{%
    basicstyle=\small\ttfamily\color{zenburn light foreground},
    keywordstyle=\bfseries\color{zenburn light keyword},
    commentstyle=\itshape\color{zenburn light comment},
    stringstyle=\color{zenburn light string},
    showstringspaces=false,
    numberstyle=\tiny\color{zenburn light line number},
}
%    \end{macrocode}
% \end{macro}
% \iffalse
%</style>
% \fi
%
% \subsubsection{Code blocks}
%
% \iffalse
%<*style>
% \fi
%
% Now we define a color box in which to place code.  The logic is that
% the |lstlisting| background frame is bad.  It is better to let a
% package that can handle frames and boxes do the real work.
% \begin{environment}{zenburn dark code block}
%    \begin{macrocode}
\RequirePackage{tcolorbox}
\newenvironment{zenburn dark code block}
{%
    \iftoggle{code block toggle}{%
        \PackageError{zenburn}{Found nested |code block|}
        {You cannot nest |code block|s}
    }{%
        \toggletrue{code block toggle}
    }
    \begin{tcolorbox}[colback=zenburn dark background,
                      colframe=zenburn dark background,
                      center]
    \lstset{style=zenburn dark style}
}{%
    \togglefalse{code block toggle}
    \end{tcolorbox}
}
%    \end{macrocode}
% \end{environment}
%
% \begin{environment}{zenburn light code block}
%    \begin{macrocode}
\newenvironment{zenburn light code block}
{%
    \iftoggle{code block toggle}{%
        \PackageError{zenburn}{Found nested |code block|}
        {You cannot nest |code block|s}
    }{%
        \toggletrue{code block toggle}
    }
    \begin{tcolorbox}[colback=zenburn light background,
                      colframe=zenburn light background,
                      center]
    \lstset{style=zenburn light style}
}{%
    \togglefalse{code block toggle}
    \end{tcolorbox}
}
%    \end{macrocode}
% \end{environment}
%
% \DescribeMacro{forced code block toggle}
% As a final step, we force a bare |lstlisting| environment into a code
% block.  This ensures the code is legible.  However, we do not want to
% override a user set code block.  For this, we need a pair of
% |\toggle|s.  The first is toggled by a user defined code block.  The
% second is used by the forced code block.
%    \begin{macrocode}
\newtoggle{code block toggle}
\newtoggle{forced code block toggle}
\RequirePackage{etoolbox}
\BeforeBeginEnvironment{lstlisting}
{%
    \iftoggle{code block toggle}{}{%
        \begin{code block}%
        \toggletrue{forced code block toggle}%
    }%
}
\AfterEndEnvironment{lstlisting}
{%
    \iftoggle{forced code block toggle}{%
        \end{code block}%
    }{}%
    \togglefalse{forced code block toggle}%
}
%    \end{macrocode}
% \iffalse
%</style>
% \fi
%
% \subsubsection{Options}
%
% \iffalse
%<*style>
% \fi
% \DescribeMacro{\zenburn@lststyle}
% \begin{macro}{\zenburn@theme}
% Now we define the options to to toggle between the dark and light
% themes.  To simplify generic code, we create a macro that is set by
% the option.  Note the trailing space in the definition.  This is
% required so that it expands correctly when used in the preamble.  We
% also define the a macro for use with |lstset| because it wants to
% remove the spaces and the main macro forces a space.  Note the lack of
% spaces in the definition.
%    \begin{macrocode}
\newcommand{\zenburn@theme}[0]{dark }
\newcommand{\zenburn@lststyle}[0]{zenburndarkstyle}
\DeclareOption{dark}{%
    \renewcommand{\zenburn@theme}[0]{dark }
    \renewcommand{\zenburn@lststyle}[0]{zenburndarkstyle}
}
\DeclareOption{light}{%
    \renewcommand{\zenburn@theme}[0]{light }
    \renewcommand{\zenburn@lststyle}[0]{zenburnlightstyle}
}
\ExecuteOptions{dark}
\DeclareOption*{\PackageWarning{zenburn}{Unknown `\CurrentOption'}}
\ProcessOptions\relax
%    \end{macrocode}
% \end{macro}
%
% \DescribeMacro{zenburn-foreground}
% \DescribeMacro{zenburn-background}
% \DescribeMacro{zenburn-comment}
% \DescribeMacro{zenburn-keyword}
% \DescribeMacro{zenburn-line-number}
% \DescribeMacro{zenburn-string}
% The last step is to define generic versions of the colors, style, and
% code block.  The colors map to the theme colors using the appropriate
% name but with the spaces replaced with hyphens.  We only define the
% theme colors that are actually used (and therefore toggled) in the
% theme.  To use a specific color, use the original name defined
% in~\cref{sec:colors}.
%    \begin{macrocode}
\colorlet{zenburn-foreground}       {zenburn \zenburn@theme foreground}
\colorlet{zenburn-background}       {zenburn \zenburn@theme background}
\colorlet{zenburn-comment}          {zenburn \zenburn@theme comment}
\colorlet{zenburn-keyword}          {zenburn \zenburn@theme keyword}
\colorlet{zenburn-line-number}      {zenburn \zenburn@theme line number}
\colorlet{zenburn-string}           {zenburn \zenburn@theme string}
%    \end{macrocode}
%
% \begin{macro}{zenburn}
% Next, we create a generic style derived from the option set when
% loading the class.
%    \begin{macrocode}
\lstdefinestyle{zenburn}{style=\zenburn@lststyle}
%    \end{macrocode}
% \end{macro}
%
% \begin{environment}{code block}
% And finally, we define a generic code block.
%    \begin{macrocode}
\newenvironment{code block}
{%
    \begin{zenburn \zenburn@theme code block}%
}{%
    \end{zenburn \zenburn@theme code block}
}
%    \end{macrocode}
% \end{environment}
%
% \iffalse
%</style>
% \fi

% \subsection{Presentation color theme}
%
% \iffalse
%<*beamercolor>
% \fi
% \begin{macro}{zenburn beamer light}
% We need a macro to flag the theme.  We cannot simply set the colors in
% the \textsf{beamer} option processing because \textsf{zenburn} will
% not be loaded yet.  We only need a flag because we have either a light
% or dark theme.
%    \begin{macrocode}
\RequirePackage{etoolbox}
\newtoggle{zenburn beamer light}
%    \end{macrocode}
% \end{macro}
%
% Next, we define the options and process them.
%    \begin{macrocode}
\DeclareOptionBeamer{light}{%
    \PassOptionsToPackage{dark}{zenburn}
    \toggletrue{zenburn beamer light}
}
\DeclareOptionBeamer{dark}{%
    \PassOptionsToPackage{light}{zenburn}
    \togglefalse{zenburn beamer light}
}
\ExecuteOptionsBeamer{light}
\DeclareOptionBeamer*{\PackageWarning{beamercolorthemezenburn}%
                      {Unknown `\CurrentOption'}}
\ProcessOptionsBeamer\relax
%    \end{macrocode}
%
% And we finish by loading the style and defining the colors.  The color
% palette will be used to set the major components of the presentation
% to give it a consistent feel.
%    \begin{macrocode}
\RequirePackage{zenburn}
\mode<presentation>
\iftoggle{zenburn beamer light}{%
    \setbeamercolor*{normal text}       {fg=zenburn light foreground,%
                                         bg=zenburn light background}
    \setbeamercolor*{structure}         {fg=zenburn background!75!brown}
    \setbeamercolor*{palette primary}   {fg=zenburn light foreground!25!brown}
    \setbeamercolor*{palette secondary} {fg=zenburn keyword}
    \setbeamercolor*{palette tertiary}  {fg=zenburn diff add}
    \setbeamercolor*{palette quaternary}{fg=zenburn light foreground}
}{%
    \PackageWarning{beamercolorthemezenburn}%
                   {`\CurrentOption' is experimental}
    \setbeamercolor*{normal text}       {fg=zenburn dark foreground,%
                                         bg=zenburn dark background}
    \setbeamercolor*{structure}         {fg=zenburn keyword}
    \setbeamercolor*{palette primary}   {fg=zenburn exception}
    \setbeamercolor*{palette secondary} {fg=zenburn comment}
    \setbeamercolor*{palette tertiary}  {fg=zenburn special char}
    \setbeamercolor*{palette quaternary}{fg=zenburn dark foreground}
}
\mode<all>
%    \end{macrocode}
%
% \iffalse
%</beamercolor>
% \fi
%
% \Finale{}
\endinput
